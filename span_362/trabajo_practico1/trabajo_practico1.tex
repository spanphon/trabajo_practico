\documentclass[12pt]{article}

\usepackage[english,spanish]{babel}
\usepackage[utf8]{inputenc}
\usepackage{t1enc}
\usepackage{graphicx}	% Put .eps and .pdf images into document

\usepackage{pstricks}
\usepackage{colortab}
\usepackage{pifont}
\usepackage{arydshln}
\usepackage{tipa}
\usepackage[numberedbib]{apacite}
\usepackage{natbib}
\usepackage{bibentry}
\usepackage{multicol}
\usepackage[colorlinks = true,
            linkcolor = magenta,
            urlcolor  = magenta,
            citecolor = magenta,
            anchorcolor = magenta]{hyperref}

\usepackage[hmargin=2.54cm,vmargin=2.54cm]{geometry}
\geometry{a4paper}
\usepackage{fancyhdr} % This should be set AFTER setting up the page geometry
\pagestyle{fancy} % options: empty , plain , fancy
\renewcommand{\headrulewidth}{0pt} % customise the layout...
\lhead{Español 362 \\ Trabajo Práctico}\chead{}\rhead{Fecha de entrega: \\11/12/2020}
\lfoot{Casillas 2019}\cfoot{\thepage}\rfoot{}


\begin{document}

\noindent Para este trabajo práctico vas a buscar a un aprendiz del español como 
L2 para enseñarle a pronunciar las oclusivas sordas (\textipa{/p, t, k/}) y 
sonoras (\textipa{/b, d, g/}) del español. Para simplificar el trabajo, vas a 
centrarte en \textipa{/d t/} en palabras que contienen estos sonidos en 
posición inicial de palabra. 
Recordad que hay dos diferencias fundamentales que tenéis que poder explicar: el 
punto de articulación y el VOT. A continuación tienes los pasos a seguir:

\begin{enumerate}
	\item \textbf{Encontrar al aprendiz}.
	\item Usando Praat, \textbf{hacer grabaciones} del participante (lista x3)
	\item \textbf{Impartir una mini clase} acerca de las oclusivas sordas y sonoras. 
	Tienes libertad absoluta con respecto a cómo se lo enseñas, pero tienes que 
	documentar el proceso. Recuerda que tu participante no tiene formación lingüística. 
	Tienes que tener cuidado con tu explicación o le vas a confundir. Ten muy claro cómo 
	lo vas a hacer de antemano. 
	\item Usando Praat, \textbf{hacer grabaciones} del participante (lista x3).
\end{enumerate}

\noindent La segunda parte consiste en el análisis de las grabaciones y tu evaluación 
del método que has utilizado. Tienes que hacer lo siguiente:

\begin{enumerate}
	\item Medir el VOT de la primera sesión (paso 2 arriba) de los sonidos dentales (\textipa{/d, t/}), 
	calcular el promedio (VOT1 + VOT2 + VOT3 / 3), sacar un gráfico representativo de cada fonema 
	(es decir, un gráfico para \textipa{/d/} y otro para \textipa{/t/}) que incluya el oscilograma, 
	el espectrograma y la segmentación del sonido.
	\item Medir el VOT de la segunda sesión (paso 4 arriba) de los sonidos dentales (\textipa{/d, t/}), 
	calcular el promedio (VOT1 + VOT2 + VOT3 / 3), sacar un gráfico representativo de cada fonema 
	(es decir, un gráfico para \textipa{/d/} y otro para \textipa{/t/}) que incluya el oscilograma, 
	el espectrograma y la segmentación del sonido.
	\item En una tabla aparte pon los valores de VOT (ver ejemplo abajo)
\end{enumerate}

\begin{table}[ht]
	\centering
	\caption{Resumen de los resultados} \vspace{.1in}
	\begin{tabular}{@{}lccccc@{}}
	\hline \\ [-2ex]
	         & \phantom{hello} & & \textipa{/d/} & & \textipa{/t/} \\ [.5ex]
	\hline \\ [-2ex]
	Sesión 1 &                 & & --            & & --            \\
	Sesión 2 &                 & & --            & & --            \\
	\hline \\ [-2ex]
	Cambio   &                 & & --            & & --            \\ [.5ex]
	\hline
	\end{tabular}
\end{table}

Por último, tienes que explicar los resultados de tu análisis. ¿Hay diferencias de VOT? 
¿A qué se deben? ¿A tu mini clase? ¿Cómo puedes estar seguro? O ¿por qué no puedes estar seguro? 
Evalúa tu método y el uso de Praat para enseñar la pronunciación. 


\clearpage

\noindent Lista de palabras (3x)

\begin{itemize}
	\item \underline{beso} es la palabra (\textipa{/\textprimstress be.so/}).
	\item \underline{peso} es la palabra (\textipa{/\textprimstress pe.so/}).
	\item \underline{dona} es la palabra (\textipa{/\textprimstress do.na/}).
	\item \underline{tona} es la palabra (\textipa{/\textprimstress to.na/}).
	\item \underline{gasa} es la palabra (\textipa{/\textprimstress ga.sa/}).
	\item \underline{casa} es la palabra (\textipa{/\textprimstress ka.sa/}).
\end{itemize}



\noindent Esquema (1 página)

\begin{enumerate}
	\item Intro
	\begin{itemize}
		\item Propósito
		\item Hipótesis
	\end{itemize}
	\item Diseño experimental
	\begin{itemize}
		\item Participante
		\item Materiales
		\item Medidas
		\item Método
	\end{itemize}
	\item Resultados
	\begin{itemize}
		\item Explicar promedios, gráficas, etc. (¿Hay diferencias de VOT?)
		\item Resumir los hallazgos
	\end{itemize}
	\item Conclusión
	\begin{itemize}
		\item Evaluación del método
		\item Evaluación de Praat
		\item Conclusión
	\end{itemize}
\end{enumerate}


\noindent Ayuda adicional

\begin{itemize}
	\item \href{https://www.jvcasillas.com/praat/slides/00_intro/index.html}{Taller 1}
	\item \href{https://www.jvcasillas.com/praat/slides/01_vot/index.html}{Taller 2}
\end{itemize}

\end{document}
